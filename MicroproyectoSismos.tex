\documentclass[12pt,letterpaper]{article}  

\usepackage[spanish]{babel} %se usa para escribir en espaol
\decimalpoint
\usepackage[T1]{fontenc} %se usa para escribir en español también xd
\usepackage[utf8]{inputenc} %se usa para escribir en español también xd
\usepackage{graphicx} %se usa para incluir imágenes
\usepackage{float} %se usa para controlar la posición de las figuras
\usepackage{wrapfig} %se usa para envolver texto alrededor de figuras y además para controlar que las imágnes no se salgan del margen al ponerlas a la izquierda o derecha
\usepackage{amsmath} %se usa para escribir ecuaciones matemáticas
\usepackage{amssymb} %se usa para escribir símbolos matemáticos
\usepackage{amsfonts} %se usa para escribir fuentes matemáticas
\usepackage{hyperref} %se usa para crear enlaces y referencias dentro del documento
\usepackage[left=2cm, right=2cm, top=2cm, bottom=3cm]{geometry} %se usa para ajustar los márgenes del documento
\usepackage{fancyhdr} %se usa para personalizar los encabezados y pies de página
\usepackage{caption} %se usa para personalizar los títulos de las figuras y tablas
\usepackage{etoolbox} %se usa para herramientas adicionales de programación
\usepackage[style=apa, citestyle=apa, language=spanish]{biblatex} %se usa para manejar bibliografías y citas
\DefineBibliographyStrings{spanish}{and = {y}}
\addbibresource{bibliografia.bib} % tu archivo .bib


\linespread{1.0}
\usepackage{setspace} %se usa para controlar el interlineado
\setlength{\parindent}{0.0in} %

\title{\textbf{Microproyecto 2}\\[0.5ex]Espectros de respuesta sismológicos}
\author{
  Oscar Eduardo García Zea$^{1}$\\
  Mar Isabel Orozco Moscoso$^{1}$\\
  Keyner Josué Paau Pop$^{1}$\\
  {\small $^{1}$Estudiantes de física, Universidad del Valle de Guatemala}
}
\date{Guatemala, \today}

\pagestyle{fancy}
\fancyhead[R]{Ecuaciones diferenciales 1}
\fancyhead[L]{}
\fancyhead[C]{}



\begin{document} %comando para iniciar el documento
\thispagestyle{fancy}
\maketitle %comando para crear el título

\section{Resolviendo la EDO}
La respuesta dinámica de un edificio ante un sismo puede modelarse como un sistema con un solo grado de libertad SDOF, cuya
ecuación diferencial está dada por:
\begin{equation*}
    m\ddot{x}(t) + b\dot{x}(t) + kx(t) = m\ddot{u_g}(t) 
\end{equation*}
donde m es la masa del edificio, b un coeficiente de amortiguamiento, k la constante de rigidez lateral, $x(t)$ el desplazamiento realtivo,
y $\ddot{u_g}(t)$ la aceleración sísmica de entrada.

La EDO homogénea asociada es:
\begin{equation*}
    m\ddot{x}(t) + b\dot{x}(t) + kx(t) = 0
\end{equation*}
y su ecuación característica es:
\begin{gather*}
    mr^2 + br + k = 0
    \Rightarrow r = \frac{-b\pm \sqrt{b^2-4mk}}{2m}
\end{gather*}
Dadas las raíces $r_1 \neq r_2$ la solución característica será:
\begin{equation*}
    x(t) = c_1e^{r_1}+c_2e^{r_2}
\end{equation*}
mientras que para $r_1 = r_2$ la solución característica será:
\begin{equation*}
x(t) = c_1e^{r}+c_2xe^{r}
\end{equation*}

\noindent\rule{\textwidth}{0.4pt}

\subsection{Ejercicio}
Dados los valores: $m = 2,000 kg$,  $b = 500 N \cdot s/m$,  $k = 80,000 N/m$:
\begin{gather*}
    \Rightarrow r = \frac{-500\pm \sqrt{500^2-4(2,000)(80,000)}}{2(2,000)} =  -0.125 \pm 6.3233i \\
    \Rightarrow x_c(t) = c_1e^{-0.125 + 6.3233i}+c_2xe^{-0.125 - 6.3233i} \\
    \Rightarrow x_c(t) = e^{-0.125t}(c_1cos(6.3233t)+c_2sen(6.3233t))
\end{gather*}

\noindent\rule{\textwidth}{0.4pt}

Para resolver la ecuación inhomogénea, se supone una entrada sísmica simplificada $\ddot{u_g}(t) = Asin(\omega t)$.\\
Utilizando variación de parámetros se propone la solución particular: $x_p = \alpha sin(\omega t) + \beta cos(\omega t)$,
\begin{gather*}
    \dot{x_p} = \alpha \omega cos(\omega t) - \beta \omega sen(\omega t), \\
    \ddot{x_p} = -\alpha \omega^2 sen(\omega t) + \beta \omega^2 cos(\omega t) \\
    \rightarrow \ddot{x}(t) + \frac{b}{m}\dot{x}(t) + \frac{k}{m}x(t) = -Asin(\omega t) \\
    \Rightarrow -\alpha \omega^2 sen(\omega t) + \beta \omega^2 cos(\omega t) + 
    \frac{b}{m}(\alpha \omega cos(\omega t) - \beta \omega sen(\omega t)) 
     + \frac{k}{m}(\alpha sin(\omega t) + \beta cos(\omega t)) = Asin(\omega t) \\
    \Rightarrow (-\alpha \omega^2 - \frac{b \beta \omega}{m}+\frac{k \alpha}{m})sen(\omega t) + 
    (\beta \omega^2+\frac{b \alpha \omega}{m}+\frac{k\beta}{m})cos(\omega t) = Asin(\omega t) \\
    \Rightarrow (\omega^2-\frac{k}{m})\alpha + \frac{b \omega}{m}\beta  = A \\
    \frac{b  \omega}{m}\alpha + (\omega^2+\frac{k}{m})\beta = 0
\end{gather*} 
Resolver este sistema de ecuaciones para $\alpha$ y $\beta$ da la solución particular.

\noindent\rule{\textwidth}{0.4pt}

\subsection{Ejercicio}
Dados los valores: $m = 2,000 kg$,  $b = 500 N \cdot s/m$,  $k = 80,000 N/m$, y la entrada sísmica $\ddot{u_g}(t)=0.3sin(3t)$:
\begin{gather*}
    \Rightarrow A = 0.3, \omega = 3 \\
    \rightarrow (3^2-\frac{80,000}{2,000})\alpha + \frac{500 \times 3}{2,000}\beta  = 0.3 \Rightarrow -31\alpha + 0.75\beta  = 0.3\\
    \rightarrow \frac{500 \times 3}{2,000}\alpha + (3^2+\frac{80,000}{2,000})\beta = 0 \Rightarrow 0.75\alpha + 49\beta = 0
\end{gather*}
Resolviendo el sistema de ecuaciones se obtiene $\alpha = -9.67384 \times 10^{-3}$ y $\beta = 1.4807\times 10^{-4}$, y la solución general
es:


\noindent\rule{\textwidth}{0.4pt}

\end{document}
