\documentclass[12pt,letterpaper]{article}  

\usepackage[spanish]{babel} %se usa para escribir en espaol
\usepackage[T1]{fontenc} %se usa para escribir en español también xd
\usepackage[utf8]{inputenc} %se usa para escribir en español también xd
\usepackage{graphicx} %se usa para incluir imágenes
\usepackage{float} %se usa para controlar la posición de las figuras
\usepackage{wrapfig} %se usa para envolver texto alrededor de figuras y además para controlar que las imágnes no se salgan del margen al ponerlas a la izquierda o derecha
\usepackage{amsmath} %se usa para escribir ecuaciones matemáticas
\usepackage{amssymb} %se usa para escribir símbolos matemáticos
\usepackage{amsfonts} %se usa para escribir fuentes matemáticas
\usepackage{hyperref} %se usa para crear enlaces y referencias dentro del documento
\usepackage[left=2cm, right=2cm, top=2cm, bottom=3cm]{geometry} %se usa para ajustar los márgenes del documento
\usepackage{fancyhdr} %se usa para personalizar los encabezados y pies de página
\usepackage{caption} %se usa para personalizar los títulos de las figuras y tablas
\usepackage{etoolbox} %se usa para herramientas adicionales de programación
\usepackage[style=apa, citestyle=apa, language=spanish]{biblatex} %se usa para manejar bibliografías y citas
\DefineBibliographyStrings{spanish}{and = {y}}
\addbibresource{bibliografia.bib} % tu archivo .bib


\linespread{1.0}
\usepackage{setspace} %se usa para controlar el interlineado
\setlength{\parindent}{0.0in} %

\title{\textbf{Microproyecto 2}\\[0.5ex]Espectros de respuesta sismológicos}
\author{
  Oscar Eduardo García Zea$^{1}$\\
  Mar Isabel Orozco Moscoso$^{1}$\\
  Keyner Josué Paau Pop$^{1}$\\
  {\small $^{1}$Estudiantes de física, Universidad del Valle de Guatemala}
}
\date{Guatemala, \today}

\pagestyle{fancy}
\fancyhead[R]{Ecuaciones diferenciales 1}
\fancyhead[L]{}
\fancyhead[C]{}



\begin{document} %comando para iniciar el documento
\thispagestyle{fancy}
\maketitle %comando para crear el título

\section{Resolviendo la EDO}
La respuesta dinámica de un edificio ante un sismo puede modelarse como un sistema con un solo grado de libertad SDOF, cuya
ecuación diferencial está dada por:
\begin{equation*}
    m\ddot{x}(t) + b\dot{x}(t) + kx(t) = m\ddot{u_g}(t) 
\end{equation*}
donde m es la masa del edificio, b un coeficiente de amortiguamiento, k la constante de rigidez lateral, $x(t)$ el desplazamiento realtivo,
y $\ddot{u_g}(t)$ la aceleración sísmica de entrada.

La EDO homogénea asociada es:
\begin{equation*}
    m\ddot{x}(t) + b\dot{x}(t) + kx(t) = 0
\end{equation*}
y su ecuación característica es:
\begin{gather*}
    mr^2 + br + k = 0
    \Rightarrow r = \frac{-b\pm \sqrt{b^2-4mk}}{2m}
\end{gather*}
Dadas las raíces $r_1 \neq r_2$ la solución característica será:
\begin{equation*}
    x(t) = c_1e^{r_1}+c_2e^{r_2}
\end{equation*}
mientras que para $r_1 = r_2$ la solución característica será:
\begin{equation*}
x(t) = c_1e^{r}+c_2xe^{r}
\end{equation*}
\end{document}
